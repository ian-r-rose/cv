%%%%%%%%%%%%%%%%%%%%%%%%%%%%%%%%%%%%%%%%%
% Classicthesis-Styled CV
% LaTeX Template
% Version 1.0 (22/2/13)
%
% This template has been downloaded from:
% http://www.LaTeXTemplates.com
%
% Original author:
% Alessandro Plasmati
%
% License:
% CC BY-NC-SA 3.0 (http://creativecommons.org/licenses/by-nc-sa/3.0/)
%
%%%%%%%%%%%%%%%%%%%%%%%%%%%%%%%%%%%%%%%%%

%----------------------------------------------------------------------------------------
%	PACKAGES AND OTHER DOCUMENT CONFIGURATIONS
%----------------------------------------------------------------------------------------

\documentclass{scrartcl}

\reversemarginpar % Move the margin to the left of the page 

\newcommand{\MarginText}[1]{\marginpar{\raggedleft\itshape\small#1}} % New command defining the margin text style

\usepackage[nochapters]{classicthesis} % Use the classicthesis style for the style of the document
\usepackage[LabelsAligned]{currvita} % Use the currvita style for the layout of the document

\renewcommand{\cvheadingfont}{\LARGE\color{Maroon}} % Font color of your name at the top

\usepackage{hyperref} % Required for adding links	and customizing them
\hypersetup{colorlinks, breaklinks, urlcolor=Maroon, linkcolor=Maroon} % Set link colors

\newlength{\datebox}\settowidth{\datebox}{Spring 2011} % Set the width of the date box in each block

\newcommand{\NewEntry}[3]{\noindent\hangindent=2em\hangafter=0 \parbox{\datebox}{\small \textit{#1}}\hspace{1.5em} #2 #3 % Define a command for each new block - change spacing and font sizes here: #1 is the left margin, #2 is the italic date field and #3 is the position/employer/location field
\vspace{0.5em}} % Add some white space after each new entry

\newcommand{\ContactInfo}[2]{\noindent \begin{flushright} {\small \textit{#1}}\hspace{1.5em} #2 \end{flushright}\vspace{-1.7em}} % Add some white space after each new entry

\newcommand{\Description}[1]{\hangindent=2em\hangafter=0\noindent\raggedright\footnotesize{#1}\par\normalsize\vspace{1em}} % Define a command for descriptions of each entry - change spacing and font sizes here

%----------------------------------------------------------------------------------------

\begin{document}

\thispagestyle{empty} % Stop the page count at the bottom of the first page

%----------------------------------------------------------------------------------------
%	NAME AND CONTACT INFORMATION SECTION
%----------------------------------------------------------------------------------------

\begin{cv}{\hspace{-4em} \spacedallcaps{Ian Rose}}
  
\vspace{0.5em}

  \hspace{-6.6em} Data science, software engineering, geology\vspace{1em} % Goal text

  \vspace{-5.4em}
  \ContactInfo{email}{\href{mailto: ian.r.rose@gmail.com}{\hspace{0.9em} ian.r.rose@gmail.com}} % Email address

  \ContactInfo{website}{\href{https://ianrose.website}{https://ianrose.website}} % Personal website

  \ContactInfo{github}{\href{https://github.com/ian-r-rose}{\hspace{5.2em} @ian-r-rose}} % Personal website

  \ContactInfo{phone}{\hspace{2.5em} +1 (510) 332-7585 } % Phone number(s)


\vspace{2em} % Extra space between major sections

%----------------------------------------------------------------------------------------
%	WORK
%----------------------------------------------------------------------------------------

\noindent\spacedallcaps{Work}\vspace{1em}

\NewEntry{2018-present}{Quansight}

\Description{\MarginText{Software Engineer}
Contractor developing tools and integrations for data scientists. Much of my work has been on improving the integration a GPU-based relational database (OmniSci) with the broader Python data science ecosystem.}

\NewEntry{2016-present}{Berkeley Institute for Data Science}

\Description{\MarginText{Postdoctoral Fellow}
Core developer for Project Jupyter. I am a developer of JupyterLab, the next-generation front-end for Jupyter notebooks. This is one of the central tools used by data scientists (primarily, but not exclusively in the Python ecosystem). In this capacity I do software engineering, mentorship, and outreach to the data science community. I am also a member of the Pangeo collaboration, which has the goal of bringing petabyte-scale analysis of Earth data to the cloud and HPC environments.}

\vspace{0.5em} % Extra space between major sections

%----------------------------------------------------------------------------------------
%	EDUCATION
%----------------------------------------------------------------------------------------
\noindent\spacedallcaps{Education}\vspace{1em}

\NewEntry{2009-2016}{The University of California, Berkeley}

\Description{\MarginText{Ph.D.}Earth and Planetary Science\newline 
  Thesis: \textit{True polar wander on convecting planets}}

\NewEntry{2005-2009}{Yale University}

\Description{\MarginText{B.S.}Geology and Physics\newline }

%------------------------------------------------

\vspace{0.5em} % Extra space between major sections

%----------------------------------------------------------------------------------------
%	COMPUTING
%----------------------------------------------------------------------------------------

\noindent\spacedallcaps{Computing}\vspace{1em}

\Description{\MarginText{Programming Languages} C, C++, JavaScript, TypeScript, Python, MATLAB/Octave, SQL, bash, awk, HTML, CSS}

\Description{\MarginText{Computational Methods} GIS analysis, visualization and mapping, ordinary/partial differential equations, regression modeling, Monte Carlo methods}

\Description{\MarginText{Software} \LaTeX, git, GitHub, node, Jupyter notebooks, standard *nix tools}

\Description{\MarginText{Operating Systems} Linux, Mac OS, Windows}


%------------------------------------------------

\vspace{0.5em} % Extra space between major sections

%----------------------------------------------------------------------------------------
%	SOFTWARE PROJECTS
%----------------------------------------------------------------------------------------

\noindent\spacedallcaps{Selected Software Projects}\vspace{1em}

\Description{\MarginText{JupyterLab} Next generation front-end for Jupyter. In addition to developing the core project, I also help develop and shepherd the extension ecosystem, including extensions for working with Dask, GitHub, Google Drive, and \LaTeX. (core developer)}

\Description{\MarginText{jupyterlab-omnisci} Integrations between JupyterLab and OmniSci, a GPU-based relational database for fast SQL analytics. (author)}
\Description{\MarginText{Interactive Earth} Educational software for teaching about the physics of planetary interiors, including thermal and thermochemical convection and seismic tomography. (author)}

\Description{\MarginText{Commuting Operation} Web application for what I want out of a real-time transit arrival service. (author)}

\Description{\MarginText{buckinghampy} Educational Python module for performing dimensional analysis. (author)}

%------------------------------------------------

%------------------------------------------------

\vspace{0.5em} % Extra space between major sections

%----------------------------------------------------------------------------------------
%	PUBLICATIONS
%----------------------------------------------------------------------------------------

\noindent\spacedallcaps{Publications}\vspace{1em}

\Description{Swanson-Hysell, N., Ramezani, J., Fairchild, L., and Rose, I.. \textit{Failed rifting and fast drifting: Midcontinent Rift development, Laurentia's rapid motion and the driver of Grenvillian orogenesis}. In Press, Geological Society of America Bulletin}

%------------------------------------------------

\Description{Rose, I. and Buffett, B.. \textit{Scaling for rates of true polar wander in convecting planets and moons}. Physics of the Earth and Planetary Interiors, Volume 273. 2017.}

%------------------------------------------------

\Description{Rose, I., Buffett, B., and Heister, T. \textit{Stability and accuracy of free surface time integration in viscous flows}. Physics of Earth and Planetary Interiors, volume 262. 2017}

%------------------------------------------------

\Description{Cottaar, S., Heister, T., Rose, I., and Unterborn, C.. \textit{BurnMan: A lower mantle mineral physics toolkit}. Geochemistry, Geophysics, Geosystems, 2014.}


\vspace{0.5em} % Extra space between major sections

%------------------------------------------------

%----------------------------------------------------------------------------------------
%	SELECTED TALKS AND CONFERENCE PROCEEDINGS
%----------------------------------------------------------------------------------------

\noindent\spacedallcaps{Selected Talks and Conference Proceedings}\vspace{1em}

%------------------------------------------------

\vspace{1em} % Extra space between major sections

\Description{Rose, I. \textit{JupyterLab}, PyData Los Angeles, 2018}
\Description{Colbert, C., and Rose, I. \textit{JupyterLab}, JupyterCon, 2018}
\Description{Colbert, C., Granger, B., and Rose, I. \textit{JupyterLab, the next-generation Jupyter frontend}, JupyterCon, 2017}
\Description{Colbert, C., Granger, B., and Rose, I. \textit{JupyterLab + Realtime Collaboration}, PyData Seattle, 2017}
\Description{Rose, I. \textit{Interactive investigations into planetary interiors}. Talk, AGU Fall Meeting 2015}
\Description{Rose, I., Buffett, B., and Heister, T. \textit{Stable time integration of a free surface in geodynamics simulations}. Poster, AGU Fall Meeting 2015}
\Description{Rose, I. \textit{True polar wander in convecting planets}. Computational Math Seminar, Clemson University, April 2014}
\Description{Cottaar, S., Heister, T., Rose, I., and Unterborn, C., \textit{An introduction to BurnMan}. Computational Infrastructure for Geodynamics Webinar, October 2015}
\Description{Rose, I., and Buffett, B.. \textit{Continents and Earth's rotational stability}. Poster, AGU Fall Meeting 2014}
\Description{Rose, I., and Buffett, B.. \textit{Rates of true polar wander in convecting planets}. Poster, SEDI meeting 2014}

%------------------------------------------------


\date{}
\end{cv}

\end{document}
